%%%%%%%%%%%%%%%%%%%%%%%%%%%%%%%%%%%%%%%%%%%%%%
%       Plantilla diseñada por José Luis Allende
%       para Universidad Andrés Bello
%       NO es una plantilla oficial,
%       basada en formato memoria de título.
%       No es necesariamente cumple con los requisitos
%       de memoria de título
%%%%%%%%%%%%%%%%%%%%%%%%%%%%%%%%%%%%%%%%%%%%%%

\documentclass[letterpaper,12pt]{report}
\usepackage[right=2cm,left=3cm,top=2cm,bottom=2cm,footskip=1.4cm]{geometry}%margenes de la pagina
\usepackage{fancyhdr}
\pagestyle{fancy}
\usepackage{tikz}
%\pagestyle{fancyplain}

% \fancyhead[C]{ UNIVERSIDAD ANDR\'ES BELLO}
% \fancyhead[R]{\thepage}
% \fancyhead[L]{}
% \fancyfoot[C]{}
\pagestyle{headings} %Se puede seleccionar esta opción o todas las anteriores.
\renewcommand{\headrulewidth}{0pt} % grosor de la línea de la cabecera

\usepackage{ucs}
\usepackage[utf8x]{inputenc}
\usepackage[spanish, es-tabla]{babel}
\usepackage[T1]{fontenc}
\usepackage{graphicx}
\usepackage{multicol} %multicolumnas
\usepackage{subfigure} %incluir multiples imágenes en una figura
% \usepackage[vlined, lined, linesnumbered, algosection]{algorithm2e} %paquete para los algoritmos en pseudocódigo
\usepackage{enumerate} %permite cambiar el item de la enumeracion mas facilmente: \begin{enumerate}[(a)] %for small alpha-characters within brackets.
						%Otra opcion renombrar el item: \renewcommand{\labelenumi}{\Alph{enumi}.}
\RequirePackage{hyperref}
\RequirePackage{url} %citacion de URL
\usepackage{hyperref}
\linespread{1.5} %interlineado

\usepackage{listings}
\lstset{language=C, numbers=left, numberstyle=\small, tabsize=2, captionpos=b, frame=single, breaklines=true, basicstyle=\normalsize, showstringspaces=false}
\renewcommand\lstlistingname{Código}

% Para algoritmos
\usepackage{algorithmic}
\usepackage{algorithm}
\floatname{algorithm}{Algoritmo}
\renewcommand{\listalgorithmname}{Lista de algoritmos}
\renewcommand{\algorithmicrequire}{\textbf{Entrada:}}
\renewcommand{\algorithmicensure}{\textbf{Retorna:}}
\renewcommand{\algorithmicprint}{\textbf{ESCRIBIR}}
\newcommand{\READ}[1]{\STATE \textbf{LEER} (#1)}
\renewcommand{\algorithmicif}{\textbf{si}}
\renewcommand{\algorithmicthen}{\textbf{entonces}}
\renewcommand{\algorithmicelse}{\textbf{sino}}
\renewcommand{\algorithmicfor}{\textbf{para}}
\renewcommand{\algorithmicforall}{\textbf{para todo}}
\renewcommand{\algorithmicdo}{\textbf{hacer}}
\renewcommand{\algorithmicrepeat}{\textbf{repetir}}
\renewcommand{\algorithmicreturn}{\textbf{retornar}}
\renewcommand{\algorithmictrue}{\textbf{verdadero}}
\renewcommand{\algorithmicfalse}{\textbf{falso}}
\renewcommand{\algorithmicend}{\textbf{fin}}
\renewcommand{\algorithmicwhile}{\textbf{mientras}}
\renewcommand{\algorithmiccomment}[1]{//#1}
\renewcommand{\algorithmicendloop}{\algorithmicend\ \algorithmicloop}
\renewcommand{\algorithmicendwhile}{\algorithmicend\ \algorithmicwhile}
\renewcommand{\algorithmicendfor}{\algorithmicend\ \algorithmicfor}
\renewcommand{\algorithmicelsif}{\algorithmicelse,\ \algorithmicif}
\renewcommand{\algorithmicendif}{\algorithmicend\ \algorithmicif}

\begin{document}
\renewcommand{\contentsname}{Tabla de Contenido}
\begin{titlepage}
\begin{center}
UNIVERSIDAD ANDRÉS BELLO\\
FACULTAD DE INGENIERÍA\\
ESCUELA DE INFORMÁTICA\\
INGENIERÍA EN COMPUTACIÓN E INFORMÁTICA
\begin{figure}[htb]
\begin{center}
	\includegraphics[width=2.97cm,bb=0 0 836 732]{color-836x732.png}
	% color-836x732.png: 836x732 pixel, 72dpi, 29.50x25.83 cm, bb=0 0 836 732
\end{center}
\end{figure}

\vspace*{0.5in}
\begin{Large}
\textbf{Informe número X o Título del Trabajo} \\
\end{Large}
\vspace*{0.3in}

% \rule{80mm}{0.1mm}\\
\vspace*{2in}

\end{center}
\begin{flushright}

\begin{tabular}{lll}
Integrantes & : & NOMBRE INTEGRANTE 1\\
            &   & RUT INTEGRANTE 1\\
            &  & NOMBRE INTEGRANTE 2\\
            &   & RUT INTEGRANTE 2\\
            &  & NOMBRE INTEGRANTE 3\\
            &   & RUT INTEGRANTE 2\\
Profesor & : & NOMBRE PROFESOR\\
Curso & : & NOMBRE DEL CURSO\\
Ayudantes & : & NOMBRE DE AYUDANTE\\
 &  & NOMBRE DE AYUDANTE\\
Fecha de entrega & : & \today
\end{tabular}
\end{flushright}
\end{titlepage}

\tableofcontents

\chapter{Primer cap\'itulo}\label{cap:1}
sdfdsfg

\section{una secci\'on}\label{sec:1}
parte que se coloca en una sección \ref{alg:sueldoNetoMensual}

\begin{algorithm}
\caption{SueldoNetoMensual}
\label{alg:sueldoNetoMensual}
\begin{algorithmic}[1] %Estenúmero dice cada cuantas líneas se enumera el algoritmo
\REQUIRE Nada
\ENSURE Nada

\STATE $hh, pago\_hora, sueldo, beneficios, descuentos, consignacion, sueldo\_neto$: Num

\PRINT (``Ingrese las horas hombre trabajadas'')
\READ {hh}
\STATE $sueldo \gets hh\times pago\_hora$
\WHILE{$cont \leq n$}
\PRINT (``Ingrese un valor'')
\READ{valor}
\IF{$cont = 1$}
\STATE $menor \gets valor$
\ELSE
\IF {$valor < menor$}
\STATE $menor\gets valor$
\ENDIF
\ENDIF
\STATE $cont \gets cont + 1$
\ENDWHILE

\PRINT (``Su sueldo neto mensual es'' + sueldo)

\end{algorithmic}
\end{algorithm}

\newpage
asdasdasd
asdasd






\end{document}

%AYUDAS:
insertar codigo desde archivo
\lstinputlisting[caption ={DESCRIPCION} , label=cod:pregunta]{SOURCE}

%insertar codigo
\begin{lstlisting}[caption={DESCRIPCION}, label=cod:pregunta]
\end{lstlisting}
